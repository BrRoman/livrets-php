\documentclass[twoside]{article}

% Géométrie de la page :
\usepackage{geometry}
\geometry{paperwidth=14.85cm, paperheight=21cm, inner=1cm, outer=1.5cm, tmargin=1cm, bmargin=1.5cm, includehead}

% Choix d'une police principale :
\usepackage{luatextra}
\setmainfont{Arno Pro}% Serveur : "Amiri".

% Langues latin et français (Césures, accents etc.) :
\usepackage[latin, frenchb]{babel}
\selectlanguage{frenchb}
\frenchbsetup{ThinColonSpace=true}

% Accès aux polices Opentype :
\usepackage{fontspec}

% Définition de polices :
\newfontfamily{\GregPlantin}[BoldFont = GregPlantin Bold,ItalicFont = GregPlantin Italic,BoldItalicFont = GregPlantin Bolditalic]{GregPlantin Regular}
\newfontfamily{\Brioso}{Brioso Pro SemiboldDisp}
\newfontfamily{\FlavGaramond}{FlavGaramond}
\newfontfamily{\Lettrines}{Plantin Std}

% Pour pouvoir insérer des expressions (et non pas seulement des valeurs) dans des commandes telles que \setlength, \addtolength, etc. :
\usepackage{calc}

% Symboles :
\usepackage{pifont}

% Textes en parallèle :
\usepackage{parallel}

% Pour augmenter l'approche des caractères :
\usepackage{soul}

% Mise en forme des titres de sections :
\usepackage[explicit]{titlesec}
\titleformat{\section}{}{}{0cm}
{
\vspace{1cm}
\fontsize{14}{16}\selectfont
\begin{flushright}
\parbox{6cm}{\centering\textbf{#1}}
\end{flushright}}
\titlespacing{\section}{0cm}{0cm}{-.8cm}

% Entêtes et pieds de pages :
\usepackage{fancyhdr}
\pagestyle{fancy}
\fancyhf{}
\cfoot{\thepage}
\lhead{\textit{Messe conventuelle}}
\rhead{\rightmark}
\renewcommand{\sectionmark}[1]{\markright{#1}}
\renewcommand{\headrulewidth}{1pt}
\renewcommand{\footrulewidth}{0pt}
\setlength{\parindent}{0cm}
\setlength{\headsep}{0.5cm} % Distance entre le header et le corps du texte.

% Textes en parallèle :
\usepackage{parallel}


%%%%%%%%%%%%%%%%%%%%%%%%%%%%%%%%%%%%
%%%%%%%%%%%%% Gregorio %%%%%%%%%%%%%
%%%%%%%%%%%%%%%%%%%%%%%%%%%%%%%%%%%%

\usepackage[autocompile]{gregoriotex}
\grechangedim{commentaryraise}{.4cm}{scalable}
\grechangestyle{modeline}{\fontsize{11}{11}\selectfont\scshape}
\grechangestyle{initial}{\fontsize{36}{36}\color{rougeliturgique}\Lettrines}
\grechangedim{afterinitialshift}{2.2mm}{fixed}
\grechangedim{beforeinitialshift}{3mm}{fixed}
\newbox\scorebox
\gresetheadercapture{commentary}{grecommentary}{string}
\let\grevanillacommentary\grecommentary
\def\grecommentary#1{\grevanillacommentary{#1\kern 0.3mm}}
\grechangestaffsize{16}% Taille des portées
\gresetbarspacing{new}
\grechangedim{bar@maior@standalone@notext}{0.3 cm}{scalable}
\grechangedim{spacebeforeeolcustos}{0.3 cm}{scalable}

% Couleur rouge :
\definecolor{rougeliturgique}{cmyk}{0.15,1,1,0}

% Style pour les références des partitions :
\grechangestyle{commentary}{\color{rougeliturgique}\itshape\fontsize{9}{8}\selectfont}

% Verset :
\renewcommand{\Vbar}{\textbf{\color{rougeliturgique}\GregPlantin\symbol{8730}}}
\catcode`\√=\active
\def√{\Vbar}

% Répons :
\renewcommand{\Rbar}{\textbf{\color{rougeliturgique}\GregPlantin\symbol{164}}}
\catcode`\®=\active
\def®{\Rbar}

% Croix haute :
\renewcommand{\GreDagger}{\textrm{\color{rougeliturgique}\FlavGaramond \symbol{8224}}}
\catcode`\†=\active
\def†{\GreDagger}

% Croix de Malte:
\catcode`\+=\active
\def+{{\fontspec{Menlo} \symbol{10016}}}

% Étoile :
\def\GreStar{{\color{rougeliturgique}\tiny\raisebox{1.5ex}{\ding{72}}}\relax}
\catcode`\*=\active
\def*{\GreStar}

% Étoile des Kyrie et Gloria étoilés :
\newcommand{\Star}{\raisebox{0.9ex}{\tiny\ding{107}}}

% Antienne :
\catcode`\ø=\active
\defø{{\fontspec{FlavGaramond} \symbol{8721}}}

% Caractères composés (æ, œ, y) + accent aigu :
\makeatletter
\def\accentaigucaractere{\makebox[0pt][c]{´}}
\newcommand\accentaigu[1]{\setlength{\@tempdima}{\widthof{#1}}\hbox{#1\kern-0.5\@tempdima\accentaigucaractere\kern0.5\@tempdima}}
\makeatother
\catcode`\ã=\active% Tilde-a
\defã{\accentaigu{æ}}
\catcode`\õ=\active% Tilde-o
\defõ{\accentaigu{œ}}
\catcode`\Ï=\active% Alt-j
\defÏ{\accentaigu{y}}

% Divers symboles liturgiques :
\catcode`\™=\active% Alt-Maj-t
\def™{\textit{\color{rougeliturgique}T.P.}}
\catcode`\∏=\active% Alt-Maj-p
\def∏{\textit{\color{rougeliturgique}Ps.}}
\catcode`\ı=\active% Alt-Maj-n
\defı{\textup{\color{rougeliturgique}N.}}

%%%%%%%%%%%%%%%%%%%%%%%%%%%%%%%%%%%%%%%%
%%%%%%%%%%%%% Fin Gregorio %%%%%%%%%%%%%
%%%%%%%%%%%%%%%%%%%%%%%%%%%%%%%%%%%%%%%%


%%%%%%%%%%%%%%%%%%%%%%%%%%%%%%%
% Commandes et environnements :
%%%%%%%%%%%%%%%%%%%%%%%%%%%%%%%

% Espace fine :
\DeclareRobustCommand{\mynobreakthinspace}{%
\leavevmode\nobreak\hspace{0.08em}}
\def~{\mynobreakthinspace{}}

% Style pour les références des partitions :
\grechangestyle{commentary}{\color{rougeliturgique}\itshape\fontsize{9}{8}\selectfont}

% Environnement Boîte (espace avant, contenu) :
\newenvironment{ParBox}[2]
{
\setlength{\parindent}{0cm}
\begin{center}
\parbox[t]{14.85cm}{\vspace{#1} #2}
\end{center}
\par}

% Style de paragraphe TitreA :
\newenvironment{TitreA}[1]
{
\fontsize{18}{13}\selectfont
\begin{center}
\MakeUppercase{#1}\end{center}}

% Style de paragraphe TitreB :
\newenvironment{TitreB}[1]
{
\vspace{.3cm}
\fontsize{12}{12}\selectfont
\textsc{#1}}

% Style de paragraphe TitreC :
\newenvironment{TitreC}[1]
{
\setlength{\parskip}{0cm}
\fontsize{10}{10}\selectfont}

% Style de paragraphe Normal :
\newenvironment{Normal}[1]
{
\setlength{\parskip}{0cm}
\fontsize{12}{14}\selectfont
\textrm{#1}}

% Commande "Dossier où se trouvent les data" :
\newcommand{\FolderData}{./Data}

% Jour (jour, rang) :
\newenvironment{JourLiturgique}[2]
{
\begin{flushright}
\parskip=.5\baselineskip
\fontsize{14}{16}\selectfont
\parbox{6cm}{\centering#1}\par
\fontsize{13}{14}\selectfont
\parbox{6cm}{\centering\textit{#2}}\par
\end{flushright}}

% Partoche :
\newenvironment{Partoche}[1]
{
\fontsize{12}{12}\selectfont
\selectlanguage{latin}
\gregorioscore{\FolderData/#1}
\selectlanguage{frenchb}
\begin{center}
\fontsize{11}{12}\selectfont
\parbox[t][\height][b]{9cm}{\vspace{-.2cm}\input{\FolderData/#1.txt}}\par\vspace{.1cm}
\end{center}}

% Tierce (partition, page psaumes) :
\newenvironment{Tierce}[2]
{
\TitreB{Office de Tierce~:}\par
\Partoche{Tierce/#1}
\vspace{-0.3cm}
\Normal{\hspace{.5cm}Psaumes~: voir livret de Tierce p. #2.}}

% Latin/francais :
\newenvironment{LatinFrancais}[2]
{
\begin{Parallel}{}{}
\selectlanguage{latin}
\ParallelLText{\fontsize{12}{13}\selectfont#1}
\selectlanguage{frenchb}
\ParallelRText{\fontsize{11}{12}\selectfont#2}
\end{Parallel}}

% Oraisons (nom de l'oraison, abréviation, jour) :
\newenvironment{Oraison}[3]
{
\TitreB{#1~:}
\LatinFrancais{\input{\FolderData/Oraisons/#3_#2_la.txt}}{\input{\FolderData/Oraisons/#3_#2_fr.txt}}}

% Préfaces (texte, chemin) :
\newenvironment{Preface}[2]
{
\TitreB{#1~:}
\LatinFrancais{\input{\FolderData/Prefaces/#2_la.txt}}{\input{\FolderData/Prefaces/#2_fr.txt}}}

% Préfaces avec nom propre (texte, chemin, nom latin au génitif - "sancti Martíni", "beátæ mártyris Agnétis" - , nom français - "saint Martin", "sainte Agnès" -) :
\newenvironment{PrefaceWithName}[4]
{
\TitreB{#1~:}
\LatinFrancais{\input{\FolderData/Prefaces/#2_la_A.txt}#3\input{\FolderData/Prefaces/#2_la_B.txt}}{\input{\FolderData/Prefaces/#2_fr_A.txt}#4\input{\FolderData/Prefaces/#2_fr_B.txt}}}

% Références :
\newenvironment{Reference}[1]
{\fontsize{9}{10}\selectfont
\hfill\parbox{6cm}{\begin{flushright}\itshape{#1}\end{flushright}}\par}

% Lectures (nom de la lecture, chemin) :
\newenvironment{Lecture}[2]
{
\TitreB{#1~:}\hfill\Reference{\input{\FolderData/Lectures/#2.ref}}
\Normal{\input{\FolderData/Lectures/#2.txt}}}

% Images (réf., commentaire) :
\newenvironment{Image}[2]
{
\newpage
\thispagestyle{empty}
\newgeometry{inner=0cm, outer=0cm, tmargin=4cm}
\begin{center}
\parbox[t]{14.85cm}{\centering\includegraphics[width=8cm]{./Data/Images/#1.jpg}\vspace{1cm}}
\parbox{7cm}{\centering{\itshape{#2}}}
\end{center}
\restoregeometry}



%

%%%%%%%%%%%%%%%%%
% Symboles spéciaux :
%%%%%%%%%%%%%%%%%

% Antienne :
\catcode`\ø=\active
\defø{{\fontspec{FlavGaramond} \symbol{8721}}}
% Répons :
\catcode`\¶=\active
\def¶{{\fontspec{FlavGaramond} \symbol{164}}}
% Verset :
\catcode`\ß=\active
\defß{{\fontspec{FlavGaramond} \symbol{8730}}}
% Croix de Malte:
\catcode`\+=\active
\def+{{\fontspec{Menlo} \symbol{10016}}}

%

%%%%%%%%%%%%%%%%%
% Hyphenations :
%%%%%%%%%%%%%%%%%

\hyphenation{al-le-lu-ia}
\hyphenation{re-gar-daient}
\hyphenation{Bé-el-zé-boul}

