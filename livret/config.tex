\documentclass[twoside]{article}

% Géométrie de la page :
\usepackage{geometry}
\geometry{paperwidth=14.85cm, paperheight=21cm, inner=1cm, outer=1.5cm, tmargin=1cm, bmargin=1.5cm, includehead}

% Choix d'une police principale :
\usepackage{luatextra}
\setmainfont{Arno Pro}% Serveur : "Amiri".

% Langues latin et français (Césures, accents etc.) :
\usepackage[latin, french]{babel}
\selectlanguage{french}
\frenchbsetup{ThinColonSpace=true}

% Accès aux polices Opentype :
\usepackage{fontspec}

% Définition de polices :
\newfontfamily{\GregPlantin}[BoldFont = GregPlantin Bold,ItalicFont = GregPlantin Italic,BoldItalicFont = GregPlantin Bolditalic]{GregPlantin Regular}
\newfontfamily{\FlavGaramond}{FlavGaramond}
\newfontfamily{\Lettrines}{Plantin Std}

% Pour pouvoir insérer des expressions (et non pas seulement des valeurs) dans des commandes telles que \setlength, \addtolength, etc. :
\usepackage{calc}

% Symboles :
\usepackage{pifont}

% Pour forcer l'usage des césures :
\pretolerance = 9000
\tolerance = 9000

% Pour augmenter l'approche des caractères :
\usepackage{soul}

% Mise en forme des titres de sections :
\usepackage[explicit]{titlesec}
\titleformat{\section}{}{}{0cm}
{
\vspace{1cm}
\fontsize{14}{16}\selectfont
\begin{flushright}
\parbox{6cm}{\centering\textbf{#1}}
\end{flushright}}
\titlespacing{\section}{0cm}{0cm}{-.8cm}

% Entêtes et pieds de pages :
\usepackage{fancyhdr}
\pagestyle{fancy}
\fancyhf{}
\cfoot{\thepage}
\lhead{\textit{Messe conventuelle}}
\rhead{\rightmark}
\renewcommand{\sectionmark}[1]{\markright{#1}}
\renewcommand{\headrulewidth}{1pt}
\renewcommand{\footrulewidth}{0pt}
\setlength{\parindent}{0cm}
\setlength{\headsep}{0.5cm} % Distance entre le header et le corps du texte.

% Textes en parallèle :
\usepackage{parallel}
\usepackage{paracol}

% Structures conditionnnelles :
\usepackage{ifthen}


%%%%%%%%%%%%%%%%%%%%%%%%%%%%%%%%%%%%
%%%%%%%%%%%%% Gregorio %%%%%%%%%%%%%
%%%%%%%%%%%%%%%%%%%%%%%%%%%%%%%%%%%%

\usepackage[autocompile]{gregoriotex}
\grechangedim{commentaryraise}{.4cm}{scalable}
\grechangestyle{modeline}{\fontsize{11}{11}\selectfont\scshape}
\grechangestyle{initial}{\fontsize{36}{36}\color{rougeliturgique}\Lettrines}
\grechangedim{afterinitialshift}{2.2mm}{fixed}
\grechangedim{beforeinitialshift}{3mm}{fixed}
\newbox\scorebox
\gresetheadercapture{commentary}{grecommentary}{string}
\let\grevanillacommentary\grecommentary
\def\grecommentary#1{\grevanillacommentary{#1\kern 0.3mm}}
\grechangestaffsize{15}% Taille des portées
\gresetbarspacing{new}
\grechangedim{bar@maior@standalone@notext}{0.3 cm}{scalable}
\grechangedim{spacebeforeeolcustos}{0.3 cm}{scalable}

% Couleur rouge :
\definecolor{rougeliturgique}{cmyk}{0.15,1,1,0}

% Style pour les références des partitions :
\grechangestyle{commentary}{\color{rougeliturgique}\itshape\fontsize{9}{8}\selectfont}

% Verset :
\renewcommand{\Vbar}{\textbf{\color{rougeliturgique}\GregPlantin\symbol{8730}}}
\catcode`\↑=\active % Alt-Maj-b
\def↑{\Vbar}

% Répons :
\renewcommand{\Rbar}{\textbf{\color{rougeliturgique}\GregPlantin\symbol{164}}}
\catcode`\®=\active % Alt-Maj-c
\def®{\Rbar}

% Croix haute :
\renewcommand{\GreDagger}{\textrm{\color{rougeliturgique}\FlavGaramond \symbol{8224}}}
\catcode`\Þ=\active % Alt-Maj-t
\defÞ{\GreDagger}

% Croix de Malte:
\catcode`\±=\active % Alt-Maj-+
\def±{{\fontspec{Menlo} \symbol{10016}}}

% Étoile dans les partitions :
\def\GreStar{{\color{rougeliturgique}\tiny\raisebox{1.5ex}{\ding{72}}}\relax}

% Étoile des Kyrie et Gloria étoilés :
\catcode`\„=\active % Alt-Maj-s
\def„{\raisebox{0.9ex}{\tiny{~\color{rougeliturgique}\ding{107}}}}

% Antienne :
\catcode`\¥=\active % Alt-Maj-g
\def¥{{\fontspec{FlavGaramond} \symbol{8721}}}

% Caractères composés (æ, œ, y) + accent aigu :
\makeatletter
\def\accentaigucaractere{\makebox[0pt][c]{´}}
\newcommand\accentaigu[1]{\setlength{\@tempdima}{\widthof{#1}}\hbox{#1\kern-0.5\@tempdima\accentaigucaractere\kern0.5\@tempdima}}
\makeatother
\catcode`\Ð=\active% Alt-Maj-h
\defÐ{\accentaigu{æ}}
\catcode`\Ï=\active% Alt-Maj-k
\defÏ{\accentaigu{œ}}
\catcode`\Ÿ=\active% Alt-Maj-y
\defŸ{\accentaigu{y}}

% Divers symboles liturgiques :
\catcode`\™=\active% Alt-Maj-8
\def™{\textit{\color{rougeliturgique}T.P.}}
\catcode`\Ø=\active% Alt-Maj-$ (dollar)
\defØ{\textit{\color{rougeliturgique}Ps.}}

%%%%%%%%%%%%%%%%%%%%%%%%%%%%%%%%%%%%%%%%
%%%%%%%%%%%%% Fin Gregorio %%%%%%%%%%%%%
%%%%%%%%%%%%%%%%%%%%%%%%%%%%%%%%%%%%%%%%


%%%%%%%%%%%%%%%%%%%%%%%%%%%%%%%
% Commandes et environnements :
%%%%%%%%%%%%%%%%%%%%%%%%%%%%%%%

% Espace fine :
\DeclareRobustCommand{\mynobreakthinspace}{%
\leavevmode\nobreak\hspace{0.08em}}
\def~{\mynobreakthinspace{}}

% Style pour les références des partitions :
\grechangestyle{commentary}{\color{rougeliturgique}\itshape\fontsize{9}{8}\selectfont}

% Environnement Boîte (espace avant, contenu) :
\newenvironment{ParBox}[2]
{
\setlength{\parindent}{0cm}
\begin{center}
\parbox[t]{14.85cm}{\vspace{#1} #2}
\end{center}
\par}

% Style de paragraphe TitreA :
\newenvironment{TitreA}[1]
{
\fontsize{18}{13}\selectfont
\begin{center}
\MakeUppercase{#1}\end{center}}

% Style de paragraphe TitreB :
\newenvironment{TitreB}[1]
{
\vspace{.3cm}
\fontsize{12}{12}\selectfont
\textsc{#1}}

% Style de paragraphe TitreC :
\newenvironment{TitreC}[1]
{
\setlength{\parskip}{0cm}
\fontsize{10}{10}\selectfont}

% Style de paragraphe Normal :
\newenvironment{Normal}[1]
{
\setlength{\parskip}{0cm}
\fontsize{12}{13}\selectfont
\textrm{#1}}

% Commande "Dossier où se trouvent les data" :
\newcommand{\FolderData}{../static/data}

% Jour (jour, rang) :
\newenvironment{JourLiturgique}[2]
{
\begin{flushright}
\parskip=.5\baselineskip
\fontsize{14}{16}\selectfont
\parbox{6cm}{\centering#1}\par
\fontsize{13}{14}\selectfont
\parbox{6cm}{\centering\textit{#2}}\par
\end{flushright}}

% Traduction:
\newenvironment{Traduction}[2]
{
\selectlanguage{french}
\fontsize{11}{12}\selectfont
\vspace{.2cm}
\leftskip#1
\rightskip\leftskip
#2\par
\vspace{.5cm}
\leftskip0cm
\rightskip\leftskip}

% Partoche :
\newenvironment{Partoche}[1]
{
\fontsize{12.2}{13}\selectfont
\selectlanguage{latin}
\gregorioscore{\FolderData/#1}
\Traduction{2cm}{\input{\FolderData/#1.txt}}}

% Tierce avec renvoi au MG (partition, page psaumes) :
\newenvironment{TierceMG}[2]
{
\TitreB{Office de Tierce~:}\par
\ifthenelse{\not\equal{#1}{}}{\Partoche{tierce/antiennes/#1}\vspace{-0.3cm}}{}
\Normal{\hspace{.5cm}Psaumes~: voir livret de Tierce inséré au début du Missel grégorien, p. #2.}}

% Tierce complet (partition, fichier psaumes) :
\newenvironment{TierceComplet}[2]
{
\TitreB{Office de Tierce~:}\par
\ifthenelse{\not\equal{#1}{}}{\Partoche{tierce/antiennes/#1}\vspace{-0.3cm}}{}
\Normal{\hspace{.5cm}\input{\FolderData/tierce/psaumes/#2.tex}}}

% Latin/francais (oraisons, préfaces…) :
\newenvironment{LatinFrancais}[2]
{
\begin{Parallel}{}{}
\selectlanguage{latin}
\ParallelLText{\fontsize{12}{13}\selectfont#1}
\selectlanguage{french}
\ParallelRText{\fontsize{11}{12}\selectfont#2}
\end{Parallel}}

% Latin/francais (1 ligne de paracol) :
\newenvironment{LigneParacol}[2]
{
\selectlanguage{latin}#1
\switchcolumn
\selectlanguage{french}#2
\switchcolumn*}

% Oraisons (nom de l'oraison, abréviation, jour) :
\newenvironment{Oraison}[3]
{
\TitreB{#1~:}
\LatinFrancais{\input{\FolderData/oraisons/#3_#2_la.txt}}{\input{\FolderData/oraisons/#3_#2_fr.txt}}}

% Préfaces (texte, chemin) :
\newenvironment{Preface}[2]
{
\TitreB{#1~:}
\LatinFrancais{\input{\FolderData/prefaces/#2_la.txt}}{\input{\FolderData/prefaces/#2_fr.txt}}}

% Préfaces avec nom propre (texte, chemin, nom latin au génitif - "sancti Martíni", "beátæ mártyris Agnétis" - , nom français - "saint Martin", "sainte Agnès" -) :
\newenvironment{PrefaceWithName}[4]
{
\TitreB{#1~:}
\LatinFrancais{\input{\FolderData/prefaces/#2_la_A.txt}#3\input{\FolderData/prefaces/#2_la_B.txt}}{\input{\FolderData/prefaces/#2_fr_A.txt}#4\input{\FolderData/prefaces/#2_fr_B.txt}}}

% Références :
\newenvironment{Reference}[1]
{\fontsize{9}{10}\selectfont
\hfill\parbox{6cm}{\begin{flushright}\itshape{#1}\end{flushright}}\par}

% Lectures (nom de la lecture, chemin) :
\newenvironment{Lecture}[2]
{
\TitreB{#1~:}\hfill\Reference{\input{\FolderData/lectures/#2.ref}}
\Normal{\input{\FolderData/lectures/#2.txt}}}

% Images (réf., largeur, espace before) :
\newenvironment{Image}[3]
{
\newpage
\thispagestyle{empty}
\newgeometry{inner=0cm, outer=0cm, tmargin=#3}
\begin{center}
\parbox[t]{14.85cm}{\centering\includegraphics[width=#2]{\FolderData/images/#1.jpg}\vspace{1cm}}
    \parbox{#2}{\centering{\itshape{\input{\FolderData/images/#1.tex}}}}
\end{center}
\restoregeometry}



%

%%%%%%%%%%%%%%%%%
% Symboles spéciaux :
%%%%%%%%%%%%%%%%%

% Antienne :
\catcode`\ø=\active
\defø{{\fontspec{FlavGaramond} \symbol{8721}}}
% Répons :
\catcode`\¶=\active
\def¶{{\fontspec{FlavGaramond} \symbol{164}}}
% Verset :
\catcode`\ß=\active
\defß{{\fontspec{FlavGaramond} \symbol{8730}}}
% Croix de Malte:
\catcode`\+=\active
\def+{{\fontspec{Menlo} \symbol{10016}}}

%

%%%%%%%%%%%%%%%%%
% Hyphenations :
%%%%%%%%%%%%%%%%%

\hyphenation{al-le-lu-ia}
\hyphenation{re-gar-daient}
\hyphenation{Bé-el-zé-boul}

