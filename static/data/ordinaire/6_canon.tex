Père infiniment bon, toi vers qui montent nos louanges, nous te supplions par Jésus Christ, ton Fils, notre Seigneur, d'accepter et de bénir ces offrandes saintes.

Nous te les présentons avant tout pour ta sainte Église catholique : accorde-lui la paix et protège-la, daigne la rassembler dans l'unité et la gouverner par toute la terre; nous les présentons en même temps pour ton serviteur le Pape N., pour notre évêque N. et tous ceux qui veillent fidèlement sur la foi catholique reçue des Apôtres.

Souviens-toi, Seigneur, de tes serviteurs (de N. et N.) et de tous ceux qui sont ici réunis, dont tu connais la foi et l'attachement. Nous t'offrons pour eux, ou ils t'offrent pour eux-mêmes et tous les leurs ce sacrifice de louange, pour leur propre rédemption, pour le salut qu'ils espèrent; et ils te rendent cet hommage, à toi, Dieu éternel vivant et vrai.

Dans la communion de toute l'Église, nous voulons nommer en premier lieu la bienheureuse Marie toujours Vierge, Mère de notre Dieu et Seigneur, Jésus Christ; saint Joseph, son époux, les saints Apôtres et Martyrs Pierre et Paul, André, Jacques et Jean, Thomas, Jacques et Philippe, Barthélemy et Matthieu, Simon et Jude, Lin, Clet, Clément, Sixte, Corneille et Cyprien, Laurent, Chrysogone, Jean et Paul, Côme et Damien et tous les saints.

Accorde-nous, par leur prière et leurs mérites, d'être, toujours et partout, forts de ton secours et de ta protection.

Voici l'offrande que nous présentons devant toi, nous, tes serviteurs, et ta famille entière dans ta bienveillance, accepte-la.

Assure toi-même la paix de notre vie, arrache-nous à la damnation et reçois-nous parmi tes élus.

Sanctifie pleinement cette offrande sur les offrandes. par la puissance de ta bénédiction, rends-la parfaite et digne de toi : qu'elle devienne pour nous le corps et le sang de ton Fils bien-aimé, Jésus Christ, notre Seigneur.

La veille de sa passion, il prit le pain dans ses mains très saintes et, les yeux levés au ciel, vers toi, Dieu, son Père tout-puissant, en te rendant grâce il le bénit, le rompit, et le donna à ses disciples, en disant :

Prenez, et mangez-en tous ceci est mon corps livré pour vous.

De même, à la fin du repas, Il prit dans ses mains cette coupe incomparable; et te rendant grâce à nouveau il la bénit, et la donna a ses disciples, en disant :

Prenez, et buvez-en tous, car ceci est la coupe de mon sang, le sang de l'Alliance nouvelle et éternelle, qui sera versé pour vous et pour la multitude en rémission des pêches.

Vous ferez cela, en mémoire de moi.

Il est grand, le mystère de la foi :
Nous proclamons ta mort, Seigneur Jésus, nous célébrons ta résurrection, nous attendons ta venue dans la gloire.

C'est pourquoi nous aussi, tes serviteurs, et ton peuple saint avec nous, faisant mémoire de la passion bienheureuse de ton Fils, Jésus Christ, notre Seigneur, de sa résurrection du séjour des morts et de sa glorieuse ascension dans le ciel, nous te présentons, Dieu de gloire et de majesté, cette offrande prélevée sur les biens que tu nous donnes, le sacrifice pur et saint, le sacrifice parfait, pain de la vie éternelle et coupe du salut.

Et comme il t'a plu d'accueillir les présents d'Abel le Juste, le sacrifice de notre père Abraham, et celui que t'offrit Melchisédech, ton grand prêtre, en signe du sacrifice parfait, regarde cette offrande avec amour et, dans ta bienveillance, accepte-la.

Nous t'en supplions, Dieu tout-puissant : qu'elle soit portée par ton ange en présence de ta gloire, sur ton autel céleste, afin qu'en recevant ici, par notre communion à l'autel, le corps et le sang de ton Fils, nous soyons comblés de ta grâce et de tes bénédictions.

Souviens-toi de tes serviteurs (de N. et N.) qui nous ont précédés, marqués du signe de la foi, et qui dorment dans la paix… Pour eux et pour tous ceux qui reposent dans le Christ, nous implorons ta bonté : qu'ils entrent dans la joie, la paix et la lumière. Et nous, pêcheurs, qui mettons notre espérance en ta miséricorde inépuisable, admets-nous dans la communauté des bienheureux Apôtres et Martyrs, de Jean Baptiste, Étienne, Matthias et Barnabé, Ignace, Alexandre, Marcellin et Pierre, Félicité et Perpétue, Agathe, Lucie, Agnès, Cécile, Anastasie, et de tous les saints. Accueille-nous dans leur compagnie, sans nous juger sur le mérite mais en accordant ton pardon, par Jésus Christ, notre Seigneur. C'est par lui que tu ne cesses de créer tous ces biens, que tu les bénis, leur donnes la vie, les sanctifies et nous en fais le don.

Par lui, avec lui et en lui, à toi, Dieu le Père tout-puissant, dans l'unité du Saint-Esprit, tout honneur et toute gloire, pour les siècles des siècles.

Amen.
