\begin{paracol}{2}

\LigneParacol
{Legem pone mihi, Dómine, viam justificatiónum tuárum~: * et exquíram eam semper.}
{Enseigne-moi, Yahweh, la voie de tes préceptes, afin que je la garde jusqu'à la fin de ma vie.}

\LigneParacol
{Da mihi intelléctum, et scrutábor legem tuam~: * et custódiam illam in toto corde meo.}
{Donne-moi l'intelligence pour que je garde ta loi, et que je l'observe de tout mon coeur.}

\LigneParacol
{Deduc me in sémitam mandatórum tuórum~: * quia ipsam vólui.}
{Conduis-moi dans le sentier de tes commandements, car j'y trouve le bonheur.}

\LigneParacol
{Inclína cor meum in testimónia tua~: * et non in avarítiam.}
{Incline mon coeur vers tes enseignements, et non vers le gain.}

\LigneParacol
{Avérte óculos meos ne vídeant vanitátem~: * in via tua vivífica me.}
{Détourne mes yeux pour qu'ils ne voient point la vanité, fais-moi vivre dans ta voie.}

\LigneParacol
{Státue servo tuo elóquium tuum, * in timóre tuo.}
{Accomplis envers ton serviteur ta promesse, que tu as faite à ceux qui te craignent.}

\LigneParacol
{Ámputa oppróbrium meum quod suspicátus sum~: * quia judícia tua jucúnda.}
{Ecarte de moi l'opprobre que je redoute, car tes préceptes sont bons.}

\LigneParacol
{Ecce, concupívi mandáta tua~: * in æquitáte tua vivífica me.}
{Je désire ardemment pratiquer tes ordonnances~: par ta justice, fais-moi vivre. }

\LigneParacol
{Et véniat super me misericórdia tua, Dómine~: * salutáre tuum secúndum elóquium tuum.}
{Que vienne sur moi ta miséricorde, Yahweh, et ton salut, selon ta parole~!}

\LigneParacol
{Et respondébo exprobrántibus mihi verbum~: * quia sperávi in sermónibus tuis.}
{Et je pourrai répondre à celui qui m'outrage, car je me confie en ta parole.}

\LigneParacol
{Et ne áuferas de ore meo verbum veritátis usquequáque~: * quia in judíciis tuis supersperávi.}
{N'ôte pas entièrement de ma bouche la parole de vérité, car j'espère en tes préceptes.}

\LigneParacol
{Et custódiam legem tuam semper~: * in sǽculum et in sǽculum sǽculi.}
{Je veux garder ta loi constamment, toujours et à perpétuité.}

\LigneParacol
{Et ambulábam in latitúdine~: * quia mandáta tua exquisívi.}
{Je marcherai au large, car je recherche tes ordonnances.}

\LigneParacol
{Et loquébar in testimóniis tuis in conspéctu regum~: * et non confundébar.}
{Je parlerai de tes enseignements devant les rois, et je n'aurai point de honte.}

\LigneParacol
{Et meditábar in mandátis tuis, * quæ diléxi.}
{Je ferai mes délices de tes commandements, car je les aime.}

\LigneParacol
{Et levávi manus meas ad mandáta tua, quæ diléxi~: * et exercébar in justificatiónibus tuis.}
{J'élèverai mes mains vers tes commandements que j'aime, et je méditerai tes lois. }

\LigneParacol
{Memor esto verbi tui servo tuo, * in quo mihi spem dedísti.}
{Souviens-toi de la parole donnée à ton serviteur, sur laquelle tu fais reposer mon espérance.}

\LigneParacol
{Hæc me consoláta est in humilitáte mea~: * quia elóquium tuum vivificávit me.}
{C'est ma consolation dans la misère, que ta parole me rende la vie.}

\LigneParacol
{Supérbi iníque agébant usquequáque~: * a lege autem tua non declinávi.}
{Des orgueilleux me prodiguent leurs railleries~: je ne m'écarte point de ta loi.}

\LigneParacol
{Memor fui judiciórum tuórum a sǽculo, Dómine~: * et consolátus sum.}
{Je pense à tes préceptes des temps passés, Yahweh, et je me console}

\LigneParacol
{Deféctio ténuit me, * pro peccatóribus derelinquéntibus legem tuam.}
{L'indignation me saisit à cause des méchants, qui abandonnent ta loi.}

\LigneParacol
{Cantábiles mihi erant justificatiónes tuæ, * in loco peregrinatiónis meæ.}
{Tes lois sont le sujet de mes cantiques, dans le lieu de mon pèlerinage.}

\LigneParacol
{Memor fui nocte nóminis tui, Dómine~: * et custodívi legem tuam.}
{La nuit je me rappelle ton nom, Yahweh, et j'observe ta loi.}

\LigneParacol
{Hæc facta est mihi~: * quia justificatiónes tuas exquisívi.}
{Voici la part qui m'est donnée~: je garde tes ordonnances.}

\end{paracol}
