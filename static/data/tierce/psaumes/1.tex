\begin{paracol}{2}

\LigneParacol
{Lucérna pédibus meis verbum tuum, * et lumen sémitis meis.}
{Ta parole est un flambeau devant mes pas, une lumière sur mon sentier.}

\LigneParacol
{Jurávi, et státui * custodíre judícia justítiæ tuæ.}
{J'ai juré, et j'y serai fidèle, d'observer les préceptes de ta justice.}

\LigneParacol
{Humiliátus sum usquequáque, Dómine~: * vivífica me secúndum verbum tuum.}
{Je suis réduit à une extrême affliction~: Yahweh, rends-moi la vie, selon ta parole.}

\LigneParacol
{Voluntária oris mei beneplácita fac, Dómine~: * et judícia tua doce me.}
{Agrée, Yahweh, l'offrande de mes lèvres, et enseigne-moi tes préceptes.}

\LigneParacol
{Ánima mea in mánibus meis semper~: * et legem tuam non sum oblítus.}
{Ma vie est continuellement dans mes mains, et je n'oublie point ta loi.}

\LigneParacol
{Posuérunt peccatóres láqueum mihi~: * et de mandátis tuis non errávi.}
{Les méchants me tendent des pièges, et je ne m'égare pas loin de tes ordonnances.}

\LigneParacol
{Hereditáte acquisívi testimónia tua in ætérnum~: * quia exsultátio cordis mei sunt.}
{J'ai tes enseignements pour toujours en héritage, car ils sont la joie de mon coeur.}

\LigneParacol
{Inclinávi cor meum ad faciéndas justificatiónes tuas in ætérnum, * propter retributiónem.}
{J'ai incliné mon coeur à observer tes lois, toujours, jusqu'à la fin. }

\LigneParacol
{Iníquos ódio hábui~: * et legem tuam diléxi.}
{Je hais les hommes au coeur double, et j'aime ta loi.}

\LigneParacol
{Adjútor et suscéptor meus es tu~: * et in verbum tuum supersperávi.}
{Tu es mon refuge et mon bouclier~; j'ai confiance en ta parole.}

\LigneParacol
{Declináte a me, malígni~: * et scrutábor mandáta Dei mei.}
{Retirez-vous de moi, méchants, et j'observerai les commandements de mon Dieu.}

\LigneParacol
{Súscipe me secúndum elóquium  tuum, et vivam~: * et non confúndas me ab exspectatióne mea.}
{Soutiens-moi selon ta promesse, afin que je vive, et ne permets pas que je sois confondu dans mon espérance.}

\LigneParacol
{Ádjuva me, et salvus ero~: * et meditábor in justificatiónibus tuis semper.}
{Sois mon appui, et je serai sauvé, et j'aurai toujours tes lois sous les yeux.}

\LigneParacol
{Sprevísti omnes discedéntes a judíciis tuis~: * quia injústa cogitátio eórum.}
{Tu méprises tous ceux qui s'écartent de tes lois, car leur ruse n'est que mensonge.}

\LigneParacol
{Prævaricántes reputávi omnes peccatóres terræ~: * ídeo diléxi testimónia tua.}
{Tu rejettes comme des scories tous les méchants de la terre~; c'est pourquoi j'aime tes enseignements.}

\LigneParacol
{Confíge timóre tuo carnes meas~: * a judíciis enim tuis tímui.}
{Ma chair frissonne de frayeur devant toi, et je redoute tes jugements. }

\LigneParacol
{Feci judícium et justítiam~: * non tradas me calumniántibus me.}
{J'observe le droit et la justice~: ne m'abandonne pas à mes oppresseurs.}

\LigneParacol
{Súscipe servum tuum in bonum~: * non calumniéntur me supérbi.}
{Prends sous ta garantie le bien de ton serviteur~; et que les orgueilleux ne m'oppriment pas~!}

\LigneParacol
{Óculi mei defecérunt in salutáre tuum~: * et in elóquium justítiæ tuæ.}
{Mes yeux languissent après ton salut, et après la promesse de ta justice.}

\LigneParacol
{Fac cum servo tuo secúndum misericórdiam tuam~: * et justificatiónes tuas doce me.}
{Agis envers ton serviteur selon ta bonté, et enseigne-moi tes lois.}

\LigneParacol
{Servus tuus sum ego~: * da mihi intelléctum, ut sciam testimónia tua.}
{Je suis ton serviteur~: donne-moi l'intelligence, pour que je connaisse tes enseignements.}

\LigneParacol
{Tempus faciéndi, Dómine~: * dissipavérunt legem tuam.}
{Il est temps pour Yahweh d'intervenir~: ils violent ta loi.}

\LigneParacol
{Ídeo diléxi mandáta tua, * super aurum et topázion.}
{C'est pourquoi j'aime tes commandements, plus que l'or et que l'or fin.}

\LigneParacol
{Proptérea ad ómnia mandáta tua dirigébar~: * omnem viam iníquam ódio hábui.}
{C'est pourquoi je trouve justes toutes tes ordonnances, je hais tout sentier de mensonge.  }

\end{paracol}
