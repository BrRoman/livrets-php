\begin{paracol}{2}

\LigneParacol
{Beátus vir, qui non ábiit in consílio impiórum, et in via peccatórum non stetit, * et in cáthedra pestiléntiæ non sedit:}
{Heureux l'homme qui ne marche pas dans le conseil des impies, qui ne se tient pas dans la voie des pécheurs et qui ne s'assied pas dans la compagnie des moqueurs, }

\LigneParacol
{Sed in lege Dómini volúntas ejus, * et in lege ejus meditábitur die ac nocte.}
{mais qui a son plaisir dans la loi de Yahweh, et qui la médite jour et nuit. }

\LigneParacol
{Et erit tamquam lignum, quod plantátum est secus decúrsus aquárum, * quod fructum suum dabit in témpore suo:}
{Il est comme un arbre planté près d'un cours d'eau, qui donne son fruit en son temps.}

\LigneParacol
{Et fólium ejus non défluet: * et ómnia quæcúmque fáciet, prosperabúntur.}
{Son feuillage ne se flétrit pas : tout ce qu'il fait réussit. }

\LigneParacol
{Non sic ímpii, non sic: * sed tamquam pulvis, quem proícit ventus a fácie terræ.}
{Il n'en est pas ainsi des impies : ils sont comme la paille que chasse le vent. }

\LigneParacol
{Ideo non resúrgent ímpii in judício: * neque peccatóres in concílio justórum.}
{Aussi les impies ne resteront-ils pas debout au jour du jugement, ni les pécheurs dans l'assemblée des justes. }

\LigneParacol
{Quóniam novit Dóminus viam justórum: * et iter impiórum períbit.}
{Car Yahweh connaît la voie du juste, mais la voie des pécheurs mène à la ruine. }

\LigneParacol
{Quare fremuérunt gentes: * et pópuli meditáti sunt inánia?}
{Pourquoi les nations s'agitent-elles en tumulte et les peuples méditent-ils de vains projets? }

\LigneParacol
{Astitérunt reges terræ, et príncipes convenérunt in unum * advérsus Dóminum, et advérsus Christum ejus.}
{Les rois de la terre se soulèvent, et les princes tiennent conseil ensemble, contre Yahweh et contre son Oint. }

\LigneParacol
{Dirumpámus víncula eórum: * et proiciámus a nobis jugum ipsórum.}
{"Brisons leurs liens, disent-ils, et jetons loin de nous leurs chaînes." }

\LigneParacol
{Qui hábitat in cælis, irridébit eos: * et Dóminus subsannábit eos.}
{Celui qui est assis dans les cieux sourit, le Seigneur se moque d'eux. }

\LigneParacol
{Tunc loquétur ad eos in ira sua, * et in furóre suo conturbábit eos.}
{Alors il leur parlera dans sa colère, et dans sa fureur il les épouvantera : }

\LigneParacol
{Ego autem constitútus sum Rex ab eo super Sion montem sanctum ejus, * prǽdicans præcéptum ejus.}
{"Et moi, j'ai été établi roi sur Sion, sa montagne sainte. Je publierai le décret :}

\LigneParacol
{Dóminus dixit ad me: * Fílius meus es tu, ego hódie génui te.}
{Yahweh m'a dit : Tu es mon Fils, je t'ai engendré aujourd'hui. }

\LigneParacol
{Póstula a me, et dabo tibi gentes hereditátem tuam, * et possessiónem tuam términos terræ.}
{Demande, et je te donnerai les nations pour héritage, pour domaine les extrémités de la terre. }

\LigneParacol
{Reges eos in virga férrea, * et tamquam vas fíguli confrínges eos.}
{Tu les conduiras avec un sceptre de fer, tu les mettras en pièces comme le vase du potier." }

\LigneParacol
{Et nunc, reges, intellégite: * erudímini, qui judicátis terram.}
{Et maintenant, rois, devenez sages ; recevez l'avertissement, juges de la terre. }

\LigneParacol
{Servíte Dómino in timóre: * et exsultáte ei cum tremóre.}
{Servez Yahweh avec crainte, tressaillez de joie avec tremblement. }

\LigneParacol
{Apprehéndite disciplínam, nequándo irascátur Dóminus, * et pereátis de via justa.}
{Recevez l'enseignement, de peur qu'il ne s'irrite et que vous ne erriez loin de la voie droite.}

\LigneParacol
{Cum exárserit in brevi ira ejus: * beáti omnes qui confídunt in eo.}
{Car bientôt s'allumera sa colère ; heureux tous ceux qui mettent en lui leur confiance. }

\LigneParacol
{Dómine, ne in furóre tuo árguas me, * neque in ira tua corrípias me.}
{Yahweh, ne me punis pas dans ta colère, et ne me châtie pas dans ta fureur. }

\LigneParacol
{Miserére mei, Dómine, quóniam infírmus sum: * sana me, Dómine, quóniam conturbáta sunt ossa mea.}
{Aie pitié de moi, Yahweh, car je suis sans force; guéris-moi, Yahweh, car mes os sont tremblants. }

\LigneParacol
{Et ánima mea turbáta est valde: * sed tu, Dómine, úsquequo?}
{Mon âme est dans un trouble extrême; et toi, Yahweh, jusques à quand? }

\LigneParacol
{Convértere, Dómine, et éripe ánimam meam: * salvum me fac propter misericórdiam tuam.}
{Reviens, Yahweh, délivre mon âme; sauve-moi à cause de ta miséricorde. }

\LigneParacol
{Quóniam non est in morte qui memor sit tui: * in inférno autem quis confitébitur tibi?}
{Car celui qui meurt n'a plus souvenir de toi; qui te louera dans le schéol? }

\LigneParacol
{Laborávi in gémitu meo, lavábo per síngulas noctes lectum meum: * lácrimis meis stratum meum rigábo.}
{Je suis épuisé à force de gémir ; chaque nuit ma couche est baignée de mes larmes, mon lit est arrosé de mes pleurs. }

\LigneParacol
{Turbátus est a furóre óculus meus: * inveterávi inter omnes inimícos meos.}
{Mon oeil est consumé par le chagrin ; j'ai vieilli parmi mes ennemis. }

\LigneParacol
{Discédite a me, omnes, qui operámini iniquitátem: * quóniam exaudívit Dóminus vocem fletus mei.}
{Éloignez-vous de moi, vous tous qui faites le mal ! Car Yahweh a entendu la voix de mes larmes. }

\LigneParacol
{Exaudívit Dóminus deprecatiónem meam, * Dóminus oratiónem meam suscépit.}
{Yahweh a exaucé ma supplication, Yahweh a accueilli ma prière. }

\LigneParacol
{Erubéscant, et conturbéntur veheménter omnes inimíci mei: * convertántur et erubéscant valde velóciter.}
{Tous mes ennemis seront confondus et saisis d'épouvante ; ils reculeront, soudain couverts de honte.  }

\end{paracol}
