\TitreC{Divisions du Psaume 9(9/10)}

\begin{paracol}{2}

\LigneParacol{0cm}
{Miserére mei, Dómine: \GreStar{} vide humilitátem meam de inimícis meis.}
{" Aie pitié de moi, Yahweh, disaient-ils; vois l'affliction où m'ont réduit mes ennemis,}

\LigneParacol{0.2cm}
{Qui exáltas me de portis mortis, \GreStar{} ut annúntiem omnes laudatiónes tuas in portis fíliæ Sion.}
{toi qui me retires des portes de la mort, afin que je puisse raconter toutes les louanges, aux portes de la fille de Sion,}

\LigneParacol{0.2cm}
{Exsultábo in salutári tuo: \GreStar{} infíxæ sunt gentes in intéritu, quem fecérunt.}
{tressaillir de joie à cause de ton salut. " Les nations sont tombées dans la fosse qu'elles ont creusée,}

\LigneParacol{0.2cm}
{In láqueo isto, quem abscondérunt, \GreStar{} comprehénsus est pes eórum.}
{dans le lacet qu'elles ont caché s'est pris leur pied.}

\LigneParacol{0.2cm}
{Cognoscétur Dóminus judícia fá\-ciens: \GreStar{} in opéribus mánuum suárum comprehénsus est peccátor.}
{Yahweh s'est montré, il a exercé le jugement, dans l'oeuvre de ses mains il a enlacé l'impie.}

\LigneParacol{0.2cm}
{Convertántur peccatóres in inférnum, \GreStar{} omnes gentes quæ obliviscúntur Deum.}
{Les impies retournent au schéol, toutes les nations qui oublient Dieu.}

\LigneParacol{0.2cm}
{Quóniam non in finem oblívio erit páuperis: \GreStar{} patiéntia páuperum non períbit in finem.}
{Car le malheureux n'est pas toujours oublié, l'espérance des affligés ne périt pas à jamais. }

\LigneParacol{0.2cm}
{Exsúrge, Dómine, non confortétur homo: \GreStar{} judicéntur gentes in conspéctu tuo.}
{Lève-toi, Yahweh ! Que l'homme ne triomphe pas ! Que les nations soient jugées devant ta face ! }

\LigneParacol{0.2cm}
{Constítue, Dómine, legislatórem super eos: \GreStar{} ut sciant gentes quóniam hómines sunt.}
{Répands sur elles l'épouvante, Yahweh ; que les peuples sachent qu'ils sont des hommes !}

\end{paracol}
\Gloria
\begin{paracol}{2}

\LigneParacol{0cm}
{Ut quid, Dómine, recessísti longe, \GreStar{} déspicis in opportunitátibus, in tribulatióne?}
{Pourquoi, Yahweh, te tiens-tu éloigné? et te caches-tu au temps de la détresse? }

\LigneParacol{0.2cm}
{Dum supérbit ímpius, incénditur pauper: \GreStar{} comprehendúntur in consíliis quibus cógitant.}
{Quand le méchant s'enorgueillit, les malheureux sont consumés; ils sont pris dans les intrigues qu'il a conçues. }

\LigneParacol{0.2cm}
{Quóniam laudátur peccátor in desidériis ánimæ suæ: \GreStar{} et iníquus benedícitur.}
{Car le méchant se glorifie de sa convoitise; le ravisseur maudit,}

\LigneParacol{0.2cm}
{Exacerbávit Dóminum peccátor, \GreStar{} secúndum multitúdinem iræ suæ non quǽret.}
{il méprise Yahweh. Dans son arrogance, le méchant dit : " Il ne punit pas ! "}

\LigneParacol{0.2cm}
{Non est Deus in conspéctu ejus: \GreStar{} inquinátæ sunt viæ illíus in omni témpore.}
{"Il n'y a pas de Dieu " : voilà toutes ses pensées. Ses voies sont prospères en tout temps!}

\LigneParacol{0.2cm}
{Auferúntur judícia tua a fácie ejus: \GreStar{} ómnium inimicórum suórum dominábitur.}
{Tes jugements sont trop élevés pour qu'il s'en inquiète ; tous ses adversaires, il les dissipe d'un souffle. }

\LigneParacol{0.2cm}
{Dixit enim in corde suo: \GreStar{} Non movébor a generatióne in generatiónem sine malo.}
{Il dit dans son coeur : " Je ne serai pas ébranlé, je suis pour toujours à l'abri du malheur. " }

\LigneParacol{0.2cm}
{Cujus maledictióne os plenum est, et amaritúdine, et dolo: \GreStar{} sub lingua ejus labor et dolor.}
{Sa bouche est pleine de malédiction, de tromperie et de violence; sous sa langue est la malice et l'iniquité. }

\LigneParacol{0.2cm}
{Sedet in insídiis cum divítibus in occúltis: \GreStar{} ut interfíciat innocéntem.}
{Il se met en embuscade près des hameaux, dans les lieux couverts il assassine l'innocent.}

\LigneParacol{0.2cm}
{Óculi ejus in páuperem respíciunt: \GreStar{} insidiátur in abscóndito, quasi leo in spelúnca sua.}
{Ses yeux épient l'homme sans défense; il est aux aguets dans le lieu couvert, comme un lion dans son fourré ;}

\LigneParacol{0.2cm}
{Insidiátur ut rápiat páuperem: \GreStar{} rápere páuperem, dum áttrahit eum.}
{il est aux aguets pour surprendre le pauvre; il se saisit du pauvre}

\LigneParacol{0.2cm}
{In láqueo suo humiliábit eum: \GreStar{} inclinábit se, et cadet, cum dominátus fúerit páuperum.}
{en le tirant dans son filet. Il se courbe, il se baisse, et les malheureux tombent dans ses griffes. }

\LigneParacol{0.2cm}
{Dixit enim in corde suo: Oblítus est Deus, \GreStar{} avértit fáciem suam ne vídeat in finem.}
{Il dit dans son coeur : " Dieu a oublié ! Il a couvert sa face, il ne voit jamais rien. " }

\end{paracol}
\Gloria
\begin{paracol}{2}

\LigneParacol{0cm}
{Exsúrge, Dómine Deus, exaltétur manus tua: \GreStar{} ne obliviscáris páuperum.}
{Lève-toi, Yahweh ; ô Dieu, lève ta main! N'oublie pas les affligés. }

\LigneParacol{0.2cm}
{Propter quid irritávit ímpius Deum? \GreStar{} dixit enim in corde suo: Non requíret.}
{Pourquoi le méchant méprise-t-il Dieu? Pourquoi dit-il en son coeur : " Tu ne punis pas? " }

\LigneParacol{0.2cm}
{Vides quóniam tu labórem et dolórem consíderas: \GreStar{} ut tradas eos in manus tuas.}
{Tu as vu pourtant; car tu regardes la peine et la souffrance; pour prendre en main leur cause.}

\LigneParacol{0.2cm}
{Tibi derelíctus est pauper: \GreStar{} órphano tu eris adjútor.}
{A toi s'abandonne le malheureux, à l'orphelin tu viens en aide. }

\LigneParacol{0.2cm}
{Cóntere brácchium peccatóris et malígni: \GreStar{} quærétur peccátum illíus, et non inveniétur.}
{Brise le bras du méchant; l'impie, - si tu cherches son crime, ne le trouveras-tu pas?}

\LigneParacol{0.2cm}
{Dóminus regnábit in ætérnum, et in sǽculum sǽculi: \GreStar{} períbitis, gentes, de terra illíus.}
{Yahweh est roi à jamais et pour l'éternité, les nations seront exterminées de sa terre. }

\LigneParacol{0.2cm}
{Desidérium páuperum exaudívit Dóminus: \GreStar{} præparatiónem cordis eórum audívit auris tua.}
{Tu as entendu le désir des affligés, Yahweh; tu affermis leur coeur, tu prêtes une oreille attentive, }

\LigneParacol{0.2cm}
{Judicáre pupíllo et húmili, \GreStar{} ut non appónat ultra magnificáre se homo super terram.}
{pour rendre justice à l'orphelin et à l'opprimé, afin que l'homme, tiré de la terre, cesse d'inspirer l'effroi. }

\end{paracol}
