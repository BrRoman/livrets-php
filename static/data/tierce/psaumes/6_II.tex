\begin{paracol}{2}

\LigneParacol
{Quóniam quis Deus præter Dóminum? * aut quis Deus præter Deum nostrum?}
{Car qui est Dieu, si ce n'est Yahweh et qui est un rocher, si ce n'est notre Dieu? }

\LigneParacol
{Deus, qui præcínxit me virtúte: * et pósuit immaculátam viam meam.}
{Le Dieu qui me ceint de force, qui rend ma voie parfaite ; }

\LigneParacol
{Qui perfécit pedes meos tamquam cervórum, * et super excélsa státuens me.}
{qui rend mes pieds semblables à ceux des biches, et me fait tenir debout sur mes hauteurs ; }

\LigneParacol
{Qui docet manus meas ad prǽlium: * et posuísti, ut arcum ǽreum, brácchia mea.}
{qui forme mes mains au combat, et mes bras tendent l'arc d'airain. }

\LigneParacol
{Et dedísti mihi protectiónem salútis tuæ: * et déxtera tua suscépit me:}
{Tu m'as donné le bouclier de ton salut, et ta droite me soutient,}

\LigneParacol
{Et disciplína tua corréxit me in finem: * et disciplína tua ipsa me docébit.}
{et ta douceur me fait grandir. }

\LigneParacol
{Dilatásti gressus meos subtus me: * et non sunt infirmáta vestígia mea:}
{Tu élargis mon pas au-dessous de moi, et mes pieds ne chancellent point. }

\LigneParacol
{Pérsequar inimícos meos et comprehéndam illos: * et non convértar, donec defíciant.}
{Je poursuis mes ennemis et je les atteins; je ne reviens pas sans les avoir anéantis. }

\LigneParacol
{Confríngam illos, nec póterunt stare: * cadent subtus pedes meos.}
{Je les brise, et ils ne se relèvent pas ; Ils tombent sous mes pieds. }

\LigneParacol
{Et præcinxísti me virtúte ad bellum: * et supplantásti insurgéntes in me subtus me.}
{Tu me ceins de force pour le combat, tu fais plier sous moi mes adversaires. }

\LigneParacol
{Et inimícos meos dedísti mihi dorsum, * et odiéntes me disperdidísti.}
{Mes ennemis !... tu leur fais tourner le dos devant moi; et j'extermine ceux qui me haïssent. }

\LigneParacol
{Clamavérunt, nec erat qui salvos fáceret ad Dóminum: * nec exaudívit eos.}
{Ils crient, et personne pour les sauver ! Ils crient vers Yahweh, et il ne leur répond pas ! }

\LigneParacol
{Et commínuam illos, ut púlverem ante fáciem venti: * ut lutum plateárum delébo eos.}
{Je les broie comme la poussière livrée au vent, je les balaie comme la boue des rues. }

\LigneParacol
{Erípies me de contradictiónibus pópuli: * constítues me in caput géntium.}
{Tu me délivres des révoltes du peuple, tu me mets à la tête des nations ;}

\LigneParacol
{Pópulus quem non cognóvi servívit mihi: * in audítu auris obedívit mihi.}
{Des peuples que je ne connaissais pas me sont asservis. Dès qu'ils ont entendu, ils m'obéissent ;}

\LigneParacol
{Fílii aliéni mentíti sunt mihi, * fílii aliéni inveteráti sunt, et claudicavérunt a sémitis suis.}
{les fils de l'étranger me flattent. Les fils de l'étranger sont défaillants, ils sortent tremblants de leurs forteresses. }

\LigneParacol
{Vivit Dóminus, et benedíctus Deus meus: * et exaltétur Deus salútis meæ.}
{Vive Yahweh et béni soit mon rocher! Que le Dieu de mon salut soit exalté; }

\LigneParacol
{Deus, qui das vindíctas mihi, et subdis pópulos sub me: * liberátor meus de inimícis meis iracúndis.}
{Dieu qui m'accorde des vengeances, qui me soumet les peuples, qui me délivre de mes ennemis !}

\LigneParacol
{Et ab insurgéntibus in me exaltábis me: * a viro iníquo erípies me.}
{Oui, tu m'élèves au-dessus de mes adversaires, tu me sauves de l'homme de violence. }

\LigneParacol
{Proptérea confitébor tibi in natiónibus, Dómine: * et nómini tuo psalmum dicam.}
{C'est pourquoi je te louerai parmi les nations, ô Yahweh; je chanterai à la gloire de ton nom :}

\LigneParacol
{Magníficans salútes Regis ejus, et fáciens misericórdiam Christo suo David: * et sémini ejus usque in sǽculum.}
{Il accorde de glorieuses délivrances à son roi, il fait miséricorde à son oint, à David et à sa postérité pour toujours. }

\end{paracol}
